% ****** Start of file apssamp.tex ******
%
%   This file is part of the APS files in the REVTeX 4.1 distribution.
%   Version 4.1r of REVTeX, August 2010
%
%   Copyright (c) 2009, 2010 The American Physical Society.
%
%   See the REVTeX 4 README file for restrictions and more information.
%
% TeX'ing this file requires that you have AMS-LaTeX 2.0 installed
% as well as the rest of the prerequisites for REVTeX 4.1
%
% See the REVTeX 4 README file
% It also requires running BibTeX. The commands are as follows:
%
%  1)  latex apssamp.tex
%  2)  bibtex apssamp
%  3)  latex apssamp.tex
%  4)  latex apssamp.tex
%
\documentclass[%
 reprint,
%superscriptaddress,
%groupedaddress,
%unsortedaddress,
%runinaddress,
%frontmatterverbose, 
%preprint,
%showpacs,preprintnumbers,
%nofootinbib,
%nobibnotes,
%bibnotes,
 amsmath,amssymb,
 aps,
%pra,
%prb,
%rmp,
%prstab,
%prstper,
%floatfix,
]{revtex4-1}

\usepackage{graphicx}% Include figure files
\usepackage{dcolumn}% Align table columns on decimal point
\usepackage{bm}% bold math
\usepackage{amsmath}
%\usepackage{hyperref}% add hypertext capabilities
%\usepackage[mathlines]{lineno}% Enable numbering of text and display math
%\linenumbers\relax % Commence numbering lines

%\usepackage[showframe,%Uncomment any one of the following lines to test 
%%scale=0.7, marginratio={1:1, 2:3}, ignoreall,% default settings
%%text={7in,10in},centering,
%%margin=1.5in,
%%total={6.5in,8.75in}, top=1.2in, left=0.9in, includefoot,
%%height=10in,a5paper,hmargin={3cm,0.8in},
%]{geometry}

\begin{document}

\preprint{APS/123-QED}

\title{Charged particles around Neutron Stars}% Force line breaks with \\
\thanks{A footnote to the article title}%

\author{Juan Luis Restrepo Lopez}
 \email{juanl.restrepo@udea.edu.co}%Lines break automatically or can be forced with \\
\author{Leonardo A. Pachon}%
 \email{leonardo.pachon@udea.edu.co}
\affiliation{%
 Instituto de Fisica, Universidad de Antioquia A.A. 1226 Medelln, Colombia
}%


\date{\today}% It is always \today, today,
             %  but any date may be explicitly specified

\begin{abstract}
An article usually includes an abstract, a concise summary of the work
covered at length in the main body of the article. 
\begin{description}
\item[Usage]
Secondary publications and information retrieval purposes.
\item[PACS numbers]
May be entered using the \verb+\pacs{#1}+ command.
\item[Structure]
You may use the \texttt{description} environment to structure your abstract;
use the optional argument of the \verb+\item+ command to give the category of each item. 
\end{description}
\end{abstract}

\pacs{Valid PACS appear here}% PACS, the Physics and Astronomy
                             % Classification Scheme.
%\keywords{Suggested keywords}%Use showkeys class option if keyword
                              %display desired
\maketitle

%\tableofcontents

\section{\label{sec:level1}Model}
Using the Weyl-Pappapetrou coordinates, with the convention ($\phi$,t,$\rho$,z),  the covariant metric tensor for the PRS solution can be written as:
\[
g_{\mu\nu}=
\left(
\begin{array}{cccc}
 \frac{\rho ^2}{f}-f \omega ^2 & f \omega  & 0 & 0 \\
 f \omega  & -f & 0 & 0 \\
 0 & 0 & \frac{e^{2 \gamma }}{f} & 0 \\
 0 & 0 & 0 & \frac{e^{2 \gamma }}{f} \\
\end{array}
\right)
\]
And the contravariant tensor is:
\[
g^{\mu\nu}=
\left(
\begin{array}{cccc}
 \frac{f}{\rho ^2} & \frac{f \omega }{\rho ^2} & 0 & 0 \\
 \frac{f \omega }{\rho ^2} & \frac{f \omega ^2}{\rho ^2}-\frac{1}{f} & 0 & 0 \\
 0 & 0 & e^{-2 \gamma } f & 0 \\
 0 & 0 & 0 & e^{-2 \gamma } f \\
\end{array}
\right)
\]
For charged particles, the super Hamiltonian can be written as:
$H=\frac{\left(\pi _{\nu }-\text{qA}_{\nu }\right) g^{\mu \nu }\left(\pi _{\mu }-\text{qA}_{\mu }\right)}{2 \mu }$

Or, explicitly:
\begin{flalign}
H  &= -\frac{(E\mu+q_pA_t(\rho,z))^2}{2\mu f_{z0}(\rho,z)} + \nonumber &\\
 &\frac{f_{z0(\rho,z)}}{2\mu} \left( \frac{-L\mu+q_pA_\phi(\rho,z)+(E\mu +q_pA_t(\rho,z))\omega_{z0}^2(\rho,z)}{\rho^2} \right) \nonumber &\\
&+ \frac{f_{z0(\rho,z)}}{2\mu}e^{-2\gamma(\rho,z)}(\pi_\rho^2+\pi_z^2)&
\end{flalign}

Where E is the energy of the orbit, $\mu$ the mass of the particle and $q_p$ its electric charge. $\pi_{\rho}$ and $\pi_z$ are the canonical moment components. The other functions, each of the coordinates $\rho$ and $z$, are the metric functions.\\

The Hamilton equations of motion are 

\begin{equation}
\frac{\partial H}{\partial q_i}=-\dot{\pi }_i\text{,  }\frac{\partial H}{\partial \pi _i}=\dot{q}_i
\end{equation}
\newline
The equation for momenta are:

\begin{flalign}
&2\mu\dot{\pi_\rho} = -\frac{2q_p(e\mu+q_pA_t(\rho,z))\partial_\rho A_t}{f_{z0}(\rho,z)} + \frac{(E\mu+q_pA_t(\rho,z))^2\partial_\rho f_{z0}(\rho,z)}{f_{z0}^2(\rho,z)} \nonumber &\\
& + \partial_\rho f_{z0} \frac{(-L\mu+q_pA_\phi(\rho,z)+(E\mu+q_pA_t(\rho,z))\omega_{z0}(\rho,z))^2}{\rho^2} \nonumber &\\
& + \partial_\rho f_{z0}e^{-2\gamma(\rho,z)}(\pi_\rho^2+\pi_z^2)+f_{z0}(\rho,z)\left( -\frac{2}{\rho^3}(L\mu-q_pA_\phi(\rho,z)\right.  \nonumber &\\
&  -(E\mu+q_pA_t(\rho,z))\omega_{z0}(\rho,z) (L\mu-q_pA_\phi(\rho,z) \nonumber &\\
&  -\omega_{z0}(\rho,z)(E\mu+q_pA_t(\rho,z)-q_p\rho\partial_\rho A_t(\rho,z))+q_p\rho\partial_\rho A_\phi(\rho,z) \nonumber &\\
& \left. +(E\mu+q_pA_t(\rho,z))\rho\partial_\rho\omega_{z0}(\rho,z))-2e^{-2\gamma(\rho,z)}(\pi_\rho^2+\pi_z^2)\partial_\rho\gamma(\rho,z)  \vphantom{\frac12}\right) \nonumber &\\
\end{flalign}

\begin{flalign}
&2\mu\dot{\pi_z} = -\frac{2q_p(e\mu+q_pA_t(\rho,z))\partial_z A_t}{f_{z0}(\rho,z)} + \frac{(E\mu+q_pA_t(\rho,z))^2\partial_z f_{z0}(\rho,z)}{f_{z0}^2(\rho,z)} \nonumber &\\
& + \partial_z f_{z0} \frac{(-L\mu+q_pA_\phi(\rho,z)+(E\mu+q_pA_t(\rho,z))\omega_{z0}(\rho,z))^2}{\rho^2} \nonumber &\\
& + \partial_z f_{z0}e^{-2\gamma(\rho,z)}(\pi_\rho^2+\pi_z^2)+f_{z0}(\rho,z)\left( -\frac{2}{\rho^2}(L\mu-q_pA_\phi(\rho,z)\right. \nonumber &\\
&  -(E\mu+q_pA_t(\rho,z))\omega_{z0}(\rho,z))(q_p\omega_{z0}(\rho,z)\partial_zA_t(\rho,z)+q_p\partial_zA_\phi(\rho,z)  \nonumber &\\
& \left. +(E\mu+q_pA_t(\rho,z)\partial_z\omega_{z0}(\rho,z)))-   2e^{-2\gamma(\rho,z)}(\pi_\rho^2+\pi_z^2)\partial_z\gamma(\rho,z) \vphantom{\frac12}\right)\nonumber &\\
\end{flalign}
%vpahtnom corrects the size of the parentesis if the height of the two lines is different

The equation for the coordinates are:
\begin{equation}
\dot{z}=\frac{1}{\mu}e^{-2\gamma(\rho,z)}f_{z0}(\rho,z)\pi_z
\end{equation}

\begin{equation}
\dot{\rho}=\frac{1}{\mu}e^{-2\gamma(\rho,z)}f_{z0}(\rho,z)\pi_\rho
\end{equation}

\subsection{Conserved quantities}
Because the field is stationary and axially symmetric there are two constants of motion: the energy and the angular momentum. This also means that the Lagrangian is independent of $t$ and $\phi$.  Taking into account that $E=\frac{\partial \mathcal{L}}{\partial \dot{t}}, \frac{L}{\mu}=\frac{\partial \mathcal{L}}{\partial \dot{\phi}}$ we get:

\begin{equation}
\begin{split}
E&=g_{t\phi}\dot{\phi}+g_{tt}\dot{t}+\frac{q_p}{m}A_t \\
\frac{L}{\mu}&=g_{t\phi}+g_{\phi\phi}\dot{\phi}+\frac{q_p}{m}A_{\phi}
\end{split}
\end{equation}

Solving for $\dot{t}$ and $\dot{\phi}$ we get:
\begin{equation}
\begin{split}
\dot{\phi}=\frac{f_{z0}(\rho,z)}{\mu\rho^2}[L-q_pA_\phi(\rho,z)-\omega_{z0}(\rho,z)(E+q_pA_t(\rho,z))]
\end{split}
\end{equation}

\begin{equation}
\begin{split}
\dot{t}&=\frac{1}{\mu f_{z0}(\rho,z)\rho^2}\{-f_{z0}^2(\rho,z)\omega_{z0}(\rho,z)[-L+q_pA_\phi(\rho,z)\\
& +\omega_{z0}(\rho,z)(E+q_pA_t(\rho,z))]+\rho^2(E+q_pA_t(\rho,z))\}
\end{split}
\end{equation}

\subsection{\label{sec:level2}Motion on the equatorial plane:}
Because we are dealing with massive particles the interval is time-like and it's true that $g_{\mu\nu}x^\mu x^\nu=-1$. Using this equation we get:
\begin{equation}
\dot{\rho}=-\frac{1}{g_{\rho\rho}}(1-g_{tt}\dot{t}^2+g_{\phi\phi}\dot{\phi}^2+2+g_{t\phi}\dot{t}\dot{\phi}+g_{zz}\dot{z}^2)
\end{equation}
For motion in the equatorial plane we take $z=0$ and, therefore, $z=\dot{z}=\ddot{z}=0$ in the last equation and in the corresponding metric functions. With this considertation, equations (8), (9) and (10) give the complete dynamics of the test particle. Taking (8) and (9) in (10) with $z=0$ we get:
\begin{equation}
\begin{split}
\dot{\rho}^2&=-\frac{1}{g_{\rho\rho}D}[D+g_{tt}(l-q_pA_\phi)^2-g_{\phi\phi}(E-q_pA_t)^2-\\
&2g_{t\phi}(E-q_pA_t)(l-q_pA_\phi)]
\end{split}
\end{equation}
Where $D=g_{\phi\phi}g_{tt}+g_{t\phi}^2$. $\dot{\rho}$ is the radial velocity of the particle, the points where $\dot{\rho}=0$ are the points where the particle is at equilibrium under the gravitational, the electromganetic interaction and the centrifugal forces. Therefore, the energy in this points is the energy of the particle when it's in this balance. This energy is the minimum energy of the particle. We call this energy the Effective Potential $V_{eff}$. Explictly, $V_{eff}$ for this metric is:
\begin{equation}
V_{eff}=q_pA_t-\frac{g_{t\phi}}{g_{\phi\phi}}(l-q_pA_\phi)\pm\frac{1}{g_{\phi\phi}}\{D[g_{\phi\phi}+(l-q_pA_{\phi})^2] \}^{\frac{1}{2}}
\end{equation}
Where $D=g_{\phi\phi}g_{tt}+g_{t\phi}^2$.\\
\vfill\eject
Using the explicit expresion for the metric functions in the equatorial plane we get:
\begin{equation}
\begin{split}
V_{eff}&=1-\dot{t}^2f_{z0}(\rho)+\left( \frac{\rho^2}{f_{z0}(\rho)}-f_{z0}(\rho)\omega_{z0}(\rho)^2 \right)\dot{\phi}^2\\
&+2f_{z0}(\rho)\omega_{z0}(\rho)\dot{t}\dot{\phi}
\end{split}
\end{equation}
Analizing the structure of $V_{eff}$ we can understand the behavior of the particles in the equatorial plane.\\
For finding if the orbit is chaotic we integrate numerically the equations of motion and plot a poincare surface. That is, we integrate the equations of motion and take the couple $(\rho,\pi_\rho)$ when $z=0$ 
\subsubsection{Wide text (A level-3 head)}
The \texttt{widetext} environment will make the text the width of the
full page, as on page~\pageref{eq:wideeq}. (Note the use the
\verb+\pageref{#1}+ command to refer to the page number.) 
\paragraph{Note (Fourth-level head is run in)}
The width-changing commands only take effect in two-column formatting. 
There is no effect if text is in a single column.

\subsection{\label{sec:citeref}Citations and References}
A citation in text uses the command \verb+\cite{#1}+ or
\verb+\onlinecite{#1}+ and refers to an entry in the bibliography. 
An entry in the bibliography is a reference to another document.

\subsubsection{Citations}
Because REV\TeX\ uses the \verb+natbib+ package of Patrick Daly, 
the entire repertoire of commands in that package are available for your document;
see the \verb+natbib+ documentation for further details. Please note that
REV\TeX\ requires version 8.31a or later of \verb+natbib+.

\paragraph{Syntax}
The argument of \verb+\cite+ may be a single \emph{key}, 
or may consist of a comma-separated list of keys.
The citation \emph{key} may contain 
letters, numbers, the dash (-) character, or the period (.) character. 
New with natbib 8.3 is an extension to the syntax that allows for 
a star (*) form and two optional arguments on the citation key itself.
The syntax of the \verb+\cite+ command is thus (informally stated)
\begin{quotation}\flushleft\leftskip1em
\verb+\cite+ \verb+{+ \emph{key} \verb+}+, or\\
\verb+\cite+ \verb+{+ \emph{optarg+key} \verb+}+, or\\
\verb+\cite+ \verb+{+ \emph{optarg+key} \verb+,+ \emph{optarg+key}\ldots \verb+}+,
\end{quotation}\noindent
where \emph{optarg+key} signifies 
\begin{quotation}\flushleft\leftskip1em
\emph{key}, or\\
\texttt{*}\emph{key}, or\\
\texttt{[}\emph{pre}\texttt{]}\emph{key}, or\\
\texttt{[}\emph{pre}\texttt{]}\texttt{[}\emph{post}\texttt{]}\emph{key}, or even\\
\texttt{*}\texttt{[}\emph{pre}\texttt{]}\texttt{[}\emph{post}\texttt{]}\emph{key}.
\end{quotation}\noindent
where \emph{pre} and \emph{post} is whatever text you wish to place 
at the beginning and end, respectively, of the bibliographic reference
(see Ref.~[\onlinecite{witten2001}] and the two under Ref.~[\onlinecite{feyn54}]).
(Keep in mind that no automatic space or punctuation is applied.)
It is highly recommended that you put the entire \emph{pre} or \emph{post} portion 
within its own set of braces, for example: 
\verb+\cite+ \verb+{+ \texttt{[} \verb+{+\emph{text}\verb+}+\texttt{]}\emph{key}\verb+}+.
The extra set of braces will keep \LaTeX\ out of trouble if your \emph{text} contains the comma (,) character.

The star (*) modifier to the \emph{key} signifies that the reference is to be 
merged with the previous reference into a single bibliographic entry, 
a common idiom in APS and AIP articles (see below, Ref.~[\onlinecite{epr}]). 
When references are merged in this way, they are separated by a semicolon instead of 
the period (full stop) that would otherwise appear.

\paragraph{Eliding repeated information}
When a reference is merged, some of its fields may be elided: for example, 
when the author matches that of the previous reference, it is omitted. 
If both author and journal match, both are omitted.
If the journal matches, but the author does not, the journal is replaced by \emph{ibid.},
as exemplified by Ref.~[\onlinecite{epr}]. 
These rules embody common editorial practice in APS and AIP journals and will only
be in effect if the markup features of the APS and AIP Bib\TeX\ styles is employed.

\paragraph{The options of the cite command itself}
Please note that optional arguments to the \emph{key} change the reference in the bibliography, 
not the citation in the body of the document. 
For the latter, use the optional arguments of the \verb+\cite+ command itself:
\verb+\cite+ \texttt{*}\allowbreak
\texttt{[}\emph{pre-cite}\texttt{]}\allowbreak
\texttt{[}\emph{post-cite}\texttt{]}\allowbreak
\verb+{+\emph{key-list}\verb+}+.

\end{document}
%
% ****** End of file apssamp.tex ******